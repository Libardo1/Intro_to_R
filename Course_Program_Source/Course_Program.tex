%     This is the LaTeX source file for the Program of the 
%     git@github.com:brfitzpatrick/Intro_to_R 
%     Short Course
%     Copyright (C) 2015  Ben R. Fitzpatrick.
%
%    This program is free software: you can redistribute it and/or modify
%    it under the terms of the GNU General Public License as published by
%    the Free Software Foundation, either version 3 of the License, or
%    (at your option) any later version.
%
%    This program is distributed in the hope that it will be useful,
%    but WITHOUT ANY WARRANTY; without even the implied warranty of
%    MERCHANTABILITY or FITNESS FOR A PARTICULAR PURPOSE.  See the
%    GNU General Public License for more details.
%
%    You should have received a copy of the GNU General Public License
%    along with this program.  If not, see <http://www.gnu.org/licenses/>.
%
%    The course author may be contacted by email at 
%    <ben.r.fitzpatrick@gmail.com>

\documentclass{article}[12pt]
\usepackage{multirow}
\usepackage{hyperref}
\usepackage{textcomp}
\usepackage{upquote}

\begin{document}

\title{Introduction to R}
\author{Ben R. Fitzpatrick\\ 
\small PhD Candidate, Mathematical Sciences School,\\
\small Queensland University of Technology, Brisbane, QLD.}
\date{\today}
\maketitle

\section*{Prior to Course}
Please do you best to install the software listed below on the laptop you will bring to the course.
Please also ensure you have `Eduroam' \url{http://www.eduroam.edu.au/} configured on your laptop so that you can access the internet during the course.  
Internet access on the laptop you are using will be vital to completing the module on version control with GitHub.
\newline
\newline
\textbf{Prior to the course please}: \begin{itemize}
\item download and install \textbf{R}
\item download and install \textbf{RStudio}
\item download and install the R packages `ggplot2' and `rgl'
\item download and install \textbf{Git}
\item create a GitHub account. \end{itemize}

\subsection*{Downloading and Installing R}
R is available for free download for Windows, MacOS, and Linux from the Comprehensive R Archive Network here: \url{http://cran.r-project.org/}.
\newline
\newline
Please also install the following R packages:  \begin{itemize}
 \item `ggplot2'
 \item `rgl' \end{itemize}
Install R packages with the `Tools' $\rightarrow$ `Install Packages' in RStudio or by executing the following code at the R command line (your computer will need to be connected to the internet to install R pacakges)o:
\begin{verbatim}
install.packages('ggplot2')
install.packages('rgl')
\end{verbatim}
You will need to selected a mirror to download the packages from Canberra or Melbourne are good choices if you are in Australia.

\subsection*{Downloading and Installing RStudio}
RStudio is available to download from here: \url{http://www.rstudio.com/}.
Please download the free, open source, desktop edition.
There is a good video introducing RStudio here \url{http://www.rstudio.com/products/rstudio/}.

\subsection*{Downloading and Installing Git}
\textbf{MS Windows} \& \textbf{Mac OS X} users please visit \url{http://git-scm.com/downloads} and follow the instructions there.\newline
\newline
\textbf{GNU+Linux} users:\\ 
Debian/Ubuntu: \begin{verbatim} sudo apt-get install git-core \end{verbatim}
Fedora/RedHat: \begin{verbatim} sudo yum install git-core \end{verbatim}

\subsection*{Creating a GitHub Account}
Navigate to \url{https://github.com/} and click the green `Sign Up' button in the top right.
\newline
\newline
\textbf{Note}: with a .edu email address you can get 5 private repositories (for now think folders to put things in) for free.
Please visit \url{https://education.github.com/discount_requests/new} to request your 5 free private repositories.
\clearpage
\section*{Program}
\begin{table}[h!]
\begin{tabular}{ p{2cm}p{7cm}p{3cm} }
\hline
\multicolumn{3}{ |c| }{\textbf{Day 1}} \\
\hline
Session                        & Topic                        & Materials \\ \hline \hline
Morning                        & Why R?                       & \begin{verbatim} Introductory_Slides \end{verbatim} \\
2.5hrs                         & First Steps with R           & \begin{verbatim} 1_Basics.R \end{verbatim} \\ \hline
\multicolumn{3}{c}{Lunch Break}  \\ \hline
Afternoon I                    & Graphics with `ggplot2'      & \begin{verbatim} 3_Intro_to_ggplot2.R \end{verbatim} \\
2.5hrs                         &                              &  \\ \hline
\multicolumn{3}{c}{Tea/Coffee/Stretch}  \\ \hline
Afternoon II                   & Linear Modelling             & \begin{verbatim} 2_Linear_Modelling.R \end{verbatim} \\ \hline
2.5hrs                         &                              &  \\ \hline
\multicolumn{3}{c}{End of Day 1}  \\ \hline
Homework                       & \multicolumn{2}{l}{Read:}  \\
                               & \multicolumn{2}{l}{\href{http://stackoverflow.com/questions/1408450/why-should-i-use-version-control}{`Why should I use version control?'} \&} \\
                               & \multicolumn{2}{l}{\href{http://stackoverflow.com/questions/2712421/r-and-version-control-for-the-solo-data-analyst}{`R and version control for the solo data analyst'}} \\ \hline \hline
\end{tabular}
\end{table}

\begin{table}[h!]
\begin{tabular}{ p{2cm}p{7cm}p{3cm} }
\hline
\multicolumn{3}{ |c| }{\textbf{Day 2}} \\
\hline
Session                       & Topic                       & Materials \\ \hline \hline
\multirow{2}{*}{Morning I}    & Programming in R            & Code File 3 \\
                              &                             &  \\ \hline
%\multicolumn{3}{c}{Tea/Coffee/Stretch}  \\ \hline
\multirow{2}{*}{Morning II}   & Programming in R            & Code File 3 \\
                              & Exercise 4 (Programming)    &  \\ \hline
\multicolumn{3}{c}{Lunch Break}  \\ \hline
\multirow{3}{*}{Afternoon I}  & Collaboration \& Version    & Link to Slides \\
                              & Control with GitHub         &  \\
                              & Excersie 5 (Collaborating with GitHub) &  \\  \hline                   
\multicolumn{3}{c}{End of Day 2}  \\ \hline
%\multirow{2}{*}{Homework}     & Continue with Collaborative & XX \\
%                              & Exercise                    & XX \\ \hline \hline
\end{tabular}
\end{table}

\begin{table}[h!]
\begin{tabular}{ p{2cm}p{7cm}p{3cm} }
\hline
\multicolumn{3}{ |c| }{\textbf{Day 3}} \\
\multirow{2}{*}{Morning }      & Collaborative Exercise      & Slides and Data \\
                              & Integrating All Skills      & \\
                              & Learned so far              & \\ \hline \hline 
\multicolumn{3}{c}{End of Course}  \\ \hline
\end{tabular}
\end{table}


%\begin{table}[h!]
%\begin{tabular}{ p{2cm}p{7cm}p{3cm} }
%\hline
%\multicolumn{3}{ |c| }{\textbf{Day 3}} \\
%\hline
%Session                       & Topic                        & Materials \\ \hline \hline
%\multirow{2}{*}{Morning I}    & Review of Results            & XX \\
%                              & Collaborative Exercise       & XX \\ \hline
%\multirow{2}{*}{Morning II}   & XX                           & XX \\
%                              & XX                           & XX \\ \hline
%\multirow{2}{*}{Afternoon I}  & 3D Graphics with 'rgl'       & XX \\
%                              & XX                           & XX \\ \hline
%\multirow{2}{*}{Afternoon II} & 3D Graphics with 'rgl'       & XX \\
%                              & Exercise 5                   & XX \\ \hline \hline
%\end{tabular}
%\end{table}

A digital version of this program along with digital versions of all other materials from this course are avavailable for download from \url{https://github.com/brfitzpatrick/Presents_Intro_to_R}.

\end{document}